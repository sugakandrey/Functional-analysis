\section{Countably-normed spaces.}
\begin{defn}
  $X$ --- linear set, $p$ --- halfnorm (function satisfying all 3 norm axioms,
  but the first one is weakened $p(x) \geq 0$, but p can be zero on non-zero
  $x$).
  \begin{enumerate}
  \item $p(x) \geq 0$
  \item $p(\lambda x) = |\lambda| p(x)$
  \item $p(x + y)$
  \end{enumerate}
\end{defn}

\begin{defn}
  $p_1, p_2, \dotsc, p_n, \dotsc$ --- halfnorms $\forall n p_n(x) = 0 \implies x
  = 0$ $(X, p_1, p_2, \dotsc, p_n, \dotsc )$ --- \textbf{countably normed space}.
\end{defn}

$x = \lim x_m \iff \forall n \in \N \lim\limits_{m \to \infty} (x_m - x ) = 0\ \|\cdot\|$
$x_m \to x^{'} \implies \forall n\ p_n(x_m - x^{'}) \to 0
x_m \to x^{''} \implies \forall n\ p_n$
If in countably-normed space we assume $p(x, y) = \sum\limits_{n = 1}^\infty
\dfrac{1}{2^n}\dfrac{p_n(x - y)}{1 + p_n(x - y)} \R^\infty$
Thus countably-normed space is always metrizable.
In countably-normed space two linear operations $(x + y, \lambda x)$ are
continuous, which means any countably normed space is also a topological vector
space.

\begin{ex}
  $C^\infty[a, b] = \Set{x(t), t \in [a, b] | x(t) \text{--- infinetely diff.}} \\
  p_n(x) = \max\limits_{[a, b]}\abs{x^{(n)}(t)}\ n = 0,1,2,\dotsc$
\end{ex}

From the next theorem we will see that $C^\infty[a, b]$ is non-normalizable (has no norm convergence
by which is equivalent to halfnorm convergence).
We will try to deduce the criteria of countably-normed space normalizability.

\begin{defn}
  System of halfnorms is called \textbf{monotonous} if $\forall x \in X,\
  \forall n \in \N\ p_n(x) \leq p_{n + 1}(X)$
\end{defn}

\begin{defn}
  ${p_n} \sim {q_n}$ if they have the same convergence (limits in both systems
  are equal).
\end{defn}

\begin{defn}
  $p_m in \{p_n\}$ --- \textbf{essential} if it can not be majorized by any of
  the preceeding halfnorms. $p_m$ can be majorized by $p_n$ if $\exists C, \forall x \in X
  p_n(x) \leq C \cdot p_m(x)$.
\end{defn}

\begin{stm}
  For any halfnorm system there exists equivalent monotonous system.
\end{stm}

\begin{proof}
  Let $q_n(x) = \sum\limits_{k = 1}^n p_k(x)$, it is obvious that every $q_n$ is
  halfnorm. ${q_n} \sim {p_n}?$
  $p_n(x_m - x) \to 0 \implies \sum\limits_{k = 1}^n p_k(x_m - x) \to 0 \implies
  q_n(x_m - x) \to 0$. Backwards proof is the same.
\end{proof}

This statement allows us to operate only on monotonous halfnorm systems.
\begin{stm}
  Two monotonous halfnorm systems are equivalent if and only if they majorize
  each other, i.e. for any halfnorm from first system there exists majorizing
  halfnorm from another and vice verca.
\end{stm}

\begin{proof}
  If two systems majorize each other obviously they are equivalent.
  Let two systems be equivalent. ${p_n}$ and ${q_n}$
\end{proof}
