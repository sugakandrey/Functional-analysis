\section{Kolmogorov theorem}
% \begin{defn}
%   $X$ --- linear set, $\tau$ --- topology on $X$, $\alpha x, x + y$ are
%   continuous on $\tau$. $E(x)$ --- vicinity $x$.
%   $\alpha x$ continuity means $\forall E(\alpha_0 x_0)\ \exists \delta > 0,\
%   \exists E(x_0) \colon |\alpha - \alpha_0| < \delta,\ x \in E(x_0) \implies
%   \alpha x \in E(\alpha_0 x_0)$ or $\alpha_0 x_0 = \lim\limits_{\alpha \to
%     \alpha _0, x \to x_0} \alpha x$
%   $x + y$ continuity means $\forall E(x_0 + y_0)\ \exists E(x_0), E(y_0) \colon
%   x \in E(x_0), y \in E(y_0) \implies x + y \in E(x_0 + y_0)$ or $x_0 + y_0 =
%   \lim\limits_{x \to x_0, y \to y_0}(x + y)$
% \end{defn}
% It is clear that normed and countably-normed spaces are just a special case of
% topological vector spaces.
% $x_0$ --- fix.

% $f(x) = x + x_0,\ X \times X $
% $f^{-1}(y) = y - x_0$
% $f$ and $f^{-1}$ are continuous (by addition continuity) --- gomeomorphism.
% $G$ --- open in $X$, $G \in \tau$, $x_0 + G = \Set{x_0 + x, x \in G}$ --- open.
% This means any topology is invariant relative to shift.
% If $\Sigma$ --- base of zero localities (union of all zero localities), any
% other zero locality 

% $x \to 0$ in $\tau$ $x + x \to 0 + 0 = 0$\\
% $\forall V \in \sigma\ \exists U \in \sigma \colon U + U \subset V,\ 2 \cdot U
% \subset U + U \subset V$ \\
% $\lambda x \to 0,\ \lambda \to 0,\ x \to 0\ \forall V \sigma\ \exists \epsilon >
% 0,\ U \in \sigma\ |\lambda| \leq \epsilon \implies \lambda U \subset V$ \\
% $\bigcup\limits_{|\lambda| \leq \epsilon} \lambda U$ --- radial, also zero
% locality. That means if $X$ is a topological vector space then system of open
% sets is invariant in relation to shift and such base can be created that:
% $\sigma$ 
% \begin{enumerate}
% \item $\forall V \in \sigma\ \exists U \in \sigma: U + U \subset V$
% \item All elements of $\sigma$ are radial, circled sets.
% \end{enumerate}

% \begin{defn}
%   Topological space is called Hausdorff space, if any couple of points can be
%   divided by their localities, i.e. there exist disjoint localities of such points.
% \end{defn}

% \begin{thm}[Kolmogorov]
%   Hausdorff topological vector space is normalizable if and only if zero has at least one 
%   limited, convex locality (the set is limited if it is absorbed by any zero's locality).
% \end{thm}

% \begin{proof}
%   Minkowski functional is halfnorm if and only if it is produced by radial,
%   circled set. If space is normalizable then unit ball is limited, convex zero's
%   locality.
%   By the former characterization of vector space topology, having limited, convex zero's
%   locality we can assume we have radial, circled locality. Produce Minkowski
%   functional using it (it will be halfnorm). We can assume the set is limited.
%   $V$ --- limited, convex, circled. $\Set{\dfrac{1}{n}V, b \in \N}$
%   $\forall$ limited. $W$ $|\lambda| \geq \lambda_0$, $V \subset \lambda W$
%   $\exists n_0 \in \N > \lambda_0$
%   $V \subset n_0 W$ $\dfrac{1}{n_0} V \subset W$ \\
%   Now we only have to check that functional is norm.
%   $p_V(x) = 0 \implies x = 0?$ \\
%   $x \in \bigcap\limits_{n = 1}^\infty (\dfrac{1}{n} V) = \{0\}$
% \end{proof}

% \begin{ex}
%   $\R^\infty = \Set{(x_1, \dotsc, x_n, \dotsc)}$ \\
%   $\Set{\bar{x} \colon x_{i_1} \in (-\delta_!, \delta_1),\ x_{i_2} \in
%     (-\delta_2, \delta_2), \dotsc, x_{i_p} \in (-\delta_p, \delta_p),\ \delta_j
%     > 0}$ --- this system will produce zero's base and insure coordinatewise
%   convergence. $P_n(\bar{x}) = |x_n|$ all of the halfnorms are essential, which
%   means the space is not normalizable.
% \end{ex}
